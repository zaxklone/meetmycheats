\documentclass[10pt,landscape]{article}
\usepackage{multicol}
\usepackage{calc}
\usepackage{ifthen}
\usepackage[landscape]{geometry}
\usepackage{hyperref}
\usepackage{color}
\usepackage[usenames,dvipsnames]{xcolor}
% To make this come out properly in landscape mode, do one of the following
% 1.
%  pdflatex latexsheet.tex
%
% 2.
%  latex latexsheet.tex
%  dvips -P pdf  -t landscape latexsheet.dvi
%  ps2pdf latexsheet.ps

%shout out Winston Chang for the guidance 
%http://www.stdout.org/∼winston/latex/

% To Do:
% \listoffigures \listoftables
% \setcounter{secnumdepth}{0}


% This sets page margins to .5 inch if using letter paper, and to 1cm
% if using A4 paper. (This probably isn't strictly necessary.)
% If using another size paper, use default 1cm margins.
\ifthenelse{\lengthtest { \paperwidth = 11in}}
	{ \geometry{top=.5in,left=.5in,right=.5in,bottom=.5in} }
	{\ifthenelse{ \lengthtest{ \paperwidth = 297mm}}
		{\geometry{top=1cm,left=1cm,right=1cm,bottom=1cm} }
		{\geometry{top=1cm,left=1cm,right=1cm,bottom=1cm} }
	}

% Turn off header and footer
\pagestyle{empty}
 

% Redefine section commands to use less space
\makeatletter
\renewcommand{\section}{\@startsection{section}{1}{0mm}%
                                {-1ex plus -.5ex minus -.2ex}%
                                {0.5ex plus .2ex}%x
                                {\normalfont\large\bfseries}}
\renewcommand{\subsection}{\@startsection{subsection}{2}{0mm}%
                                {-1explus -.5ex minus -.2ex}%
                                {0.5ex plus .2ex}%
                                {\normalfont\normalsize\bfseries}}
\renewcommand{\subsubsection}{\@startsection{subsubsection}{3}{0mm}%
                                {-1ex plus -.5ex minus -.2ex}%
                                {1ex plus .2ex}%
                                {\normalfont\small\bfseries}}
\makeatother

% Define BibTeX command
\def\BibTeX{{\rm B\kern-.05em{\sc i\kern-.025em b}\kern-.08em
    T\kern-.1667em\lower.7ex\hbox{E}\kern-.125emX}}

% Don't print section numbers
\setcounter{secnumdepth}{0}


\setlength{\parindent}{0pt}
\setlength{\parskip}{0pt plus 0.5ex}


% -----------------------------------------------------------------------

\begin{document}

\raggedright
\footnotesize


\begin{center}
     \Large{\textbf{ My Everyday Cheat Sheet}} \\
\end{center}

%uncomment when the cheat sheet grows
\begin{multicols}{1}


% multicol parameters
% These lengths are set only within the two main columns
%\setlength{\columnseprule}{0.25pt}
\setlength{\premulticols}{1pt}
\setlength{\postmulticols}{1pt}
\setlength{\multicolsep}{1pt}
\setlength{\columnsep}{2pt}


\section{VIM Startups}
\begin{tabular}{@{}ll@{}}
\texttt{vim -p[N]}    & \colorbox{Aquamarine}{Open N tab pages.} \\
\texttt{vim } \textit{file1 file2}  & Opens up files sequentially \\
\texttt{vim  +} \textit{file}  & Open file at last line\\
\texttt{vim  +n } \textit{file}  & Open file at line number n\\
\texttt{vim  +/pattern} \textit{file}  & Open file at \texttt{pattern}\\
\texttt{vim  -t } tag  & look up tag and start editing at its defintion\\
\texttt{view}  \textit{file}  & Open up \textit{file} in read-only mode\\
\end{tabular}


\section{quick and dirty git}
\begin{tabular}{@{}ll@{}}
\texttt{git clone } \textit{repo} & clones repo in cur dir\\
\texttt{git status }  & status of current repo, what has changed\\
\texttt{git init }  & initializes a repo in current dir\\
\texttt{git config user.name}  \textit{username} & set username\\
\texttt{git config user.email} \textit{username@email.com} & set email\\
\end{tabular}



\section{i3 display hacks}
\newlength{\MyLen}
\settowidth{\MyLen}{\texttt{xrandr --output VGA1 --auto --right-of LVDS1} \ }
\begin{tabular}{@{}p{\the\MyLen}%
                @{}p{ \linewidth-\the\MyLen}@{}}
\texttt{xrandr --output VGA1 --auto --right-of LVDS1} & extend 2nd display to right of 1st display \\
\texttt{xrandr --output VGA1 --auto --left-of LVDS1} & extend 2nd display to left of 1st display \\
\end{tabular}


\section{i3 quick and dirty}
\begin{tabular}{@{}ll@{}}
\texttt{MOD-f} & enter fullscreen mode\\
\texttt{MOD-d} & access dmenu\\
\texttt{MOD-s} & stacked layout\\
\texttt{MOD-a} & focus parent\\
\texttt{MOD-e} & splith/splitv it toggles\\
\end{tabular}

\section{Ctags}
\settowidth{\MyLen}{\texttt{:set tags=tags} }
\begin{tabular}{@{}p{\the\MyLen}%
                @{}p{\linewidth-\the\MyLen}@{}}
%\begin{tabular}{@{}ll@{}}
\texttt{ctags -R}  &  create tags file in current directory \\
\texttt{:set tags=tags}   &  Set tags of tag file in current directory in \textit{VIM} \\
\end{tabular}



\section{Python Commandline hacks}
\settowidth{\MyLen}{\texttt{python -m SimpleHTTPServer} }
\begin{tabular}{@{}p{\the\MyLen}%
                @{}p{\linewidth-\the\MyLen}@{}}
\texttt{python -m SimpleHTTPServer} & creates a webserver on port 8000 ./ as root dir  \\
\end{tabular}
\end{multicols}



\begin{center}

\rule{0.3\linewidth}{0.25pt}
\scriptsize

zaxklone meetmycheats

\href{https://github.com/zaxklone/meetmycheats}{https://github.com/zaxklone/meetmycheats}
\end{center}



\end{document}
